\documentclass[journal, a4paper]{IEEEtran}

\usepackage{graphicx}  

\usepackage{url}  

\usepackage{amsmath}    

\usepackage{graphicx}
\usepackage{subfigure}
\usepackage{wrapfig}
\usepackage{xeCJK}
\setCJKmainfont[
  BoldFont=WenQuanYi Zen Hei,
  ItalicFont=AR PL KaitiM GB]
  {AR PL SungtiL GB}
\setCJKsansfont{WenQuanYi Zen Hei}
\setCJKmonofont{cwTeXFangSong}

\begin{document}

    \title{基于特征匹配与动态规划的视频防抖算法}
    \author{刘璐 13307130520}
    \maketitle

\section{引言}
    随着摄影录像技术的发展,越来越多的人选择使用小型手持设备进行视频录制,记录下一些难忘的时刻。大部分使用小型手持设备进行视频录制的使用者在录制过程中一方面缺乏固定镜头的设备,另一方面没有专业的录像技术,以致于使用小型手持设备录制的视频都会出现一定的画面抖动,在一定程度上影响了观影效果。视频防抖技术的出现即是为了解决这一难题。 \\
    
    要减少视频中影响视频质量的抖动现象,一种方法是使用手持设备的陀螺仪、加速度仪,记录视频拍摄时设备在三维空间三个角度的旋转以及加速度,从设备的移动路径反向恢复视频的质量。这样的方法能获取录制设备的移动轨迹,从而提高视频处理的效果\cite{GYR1}\cite{GYR2}。但是,一方面,对于网络上大部分已存在的视频来说,录制时的设备移动记录已经无法获取;另一方面,不同的设备内置的陀螺仪、加速度仪会有不同的误差,要设计对于所有设备都行之有效的算法有一定的难度。\\
    
    另一种方法是,直接对视频进行处理。传统的视频防抖方法都是基于2D图像进行处理的。一般的处理方法为对相邻的帧进行特征点匹配,从帧与帧之间的对齐关系获得录像设备视角的移动,再对视角的移动路线进行平滑处理,使得输出视频的移动相对稳定\cite{L1Opt}。也有新兴的技术通过视频重构三维模型,来实现视频防抖。该技术重构像素点的3D点云,以及录影设备在三维空间上的移动路线,对设备移动路线进行平滑处理后,利用3D点云恢复视频图像\cite{FLiu1}\cite{FLiu2}。 \\
    
    我的方法主要是基于2D图像进行处理的。在对相机移动线路进行平滑处理时,只考虑线性的移动,利用若干条线段近似原始的相机移动路线。在拟合过程中,选定若干关键帧为不动帧,从一个关键帧到另一个关键帧的过程中,规划的相机路径始终采用相同的变换进行移动。在关键帧选取的关键环节,采用动态规划算法,最优化关键帧的选取,从而获得尽量与原始相机移动路径相似的线性移动路径,实现视频的防抖。

\section{视频防抖算法}
\subsection{固定视角的视频防抖}
    刚开始的时候,我采用了Matlab手册中基于特征点匹配的视频防抖方法\cite{Matlab}。首先,对于视频中的每一帧图像$I_{1}, I_{2}, \cdots, I_{n}$,求SURF特征点。然后,对视频中相邻两帧的图像$(I_{t}, I_{t+1})$进行特征点匹配,获得相邻两帧的变换矩阵$F_{t}$。相机路径$C_{t}$可以与前一帧的相机路径$C_{t-1}$以及变换矩阵$F_{t}$相关:$C_{t}=C_{t-1}*F_{t}$,且对$C_{t}$进行迭代计算可得到$C_{t}=F_{t}F_{t-1}\cdots F_{1}$。 \\
    
    进行视频防抖最直接的想法就是:对每一帧原始图像,使用$C_{t}$矩阵进行变换,这样一来画面中的大部分特征点都会基本保持静止,从而实现了减少视频抖动的目的。这样的方法在固定视角、单一场景的视频中的确能取得较好的效果。但是,在一个完整视频中,往往会有相机的移动、场景的切换。因为该方法的所有帧都是向第一帧对齐的,后面的帧中特征点基本与第一帧中特征点的位置吻合,所以在遇到相机移动和场景切换的时候,相机的拍摄内容已经移出或部分移出第一帧的取景框内,这个时候在第一帧的取景框内的内容即为缺省值。因此,该方法对有场景切换的视频是基本无效的, 如图。
    
\subsection{选取固定关键帧的视频防抖}
    考虑到上述方法失效的原因主要是原始相机路径的移动与输出视频的取景框固定不动的矛盾。所以,我开始寻求让取景框动起来的视频防抖方法。 \\
    
    在这个过程中,我读到了M. Grundmann提出的利用L1范数线性优化相机路径的方法\cite{L1Opt}。在该论文中提到,在专业的视频拍摄中,相机的移动路线$P(t)$主要分为以下三种:
    \begin{itemize}
    \item 固定不动的相机路线,$DP(t)=0$
    \item 以恒定速度移动的相机路线,$DP^2(t)=0$
    \item 以恒定加速度移动的相机路线,$DP^3(t)=0$
    \end{itemize}
    在该论文中,M. Grundmann旨在用上述三种相机移动路径来近似原始的相机路径,使得输出的视频结果即保持了原来的视频内容,又实现了防抖的效果。在该论文中,M. Grundmann最终采用了L1范数线性规划的方法进行路径优化。\\
    
    固定不动的相机路线的视频防抖问题在上一小节中已经完美解决。我认为,长时间以恒定加速度移动的相机路线只会在有专业的控制相机移动路线的设备上才有可能获得。一般在使用手持设备进行拍摄的条件下,使用者在移动相机时的目的在于获得恒定速度移动的相机路线,在这种情况下,恒定加速度移动的相机路线只会在使用者希望相机改变移动速度的短暂时刻出现。所以,我决定只采用恒定速度移动的相机路线去近似原始输入视频的相机路线。\\
    
   一开始,我采用了固定间隔的关键帧选取,如图\ref{fig:naive_motion}。即每$k(k=30)$帧选一帧作为关键帧,关键帧的相机路径与路径规划后的相机路径$P_t$吻合,$P_t=C_t$,并假设取景框在关键帧$C_{ki}$与$C_{ki+k}$之间以恒定的速度移动,即变换矩阵相同,
   $$H_{ki+1}=H_{ki+2}=\cdots=H_{ki+k}\eqno{(1)}$$同时由于关键帧的性质,应有
   $$H_{ki+1}*H_{ki+2}*\cdots*H_{ki+k}=F_{ki+1}*F_{ki+2}*\cdots*F_{ki+k}\eqno{(2)}$$
   由等式($1$)($2$)得:$$H_{ki+j}=(F_{ki+1}*F_{ki+2}*\cdots*F_{ki+k})^{\frac{1}{k}}\eqno{(3)}$$从而可以求得每一帧图像的变换矩阵
   $$T_{ki+j}=H_{ki+j}^{-j}*F_{ki+1}*F_{ki+2}*\cdots*F_{ki+j}\eqno{(4)}$$
   这样一来,以固定间隔取关键帧,关键帧间的相邻帧的取景窗以相同的变换矩阵进行变换,就实现了在两个关键帧之间的画面以比较平滑的方式进行移动、变换。\\
   \begin{figure}[!hbt]
        \begin{center}
        \includegraphics[width=\columnwidth]{naive_motion.png}
        \caption{蓝色的折线是视频原始的相机在$x$方向的移动,橙色的折线是每$k(k=30)$帧取一帧作为关键帧,用关键帧之间的折线去近似相机移动的结果。可以看到,在第$30-60,150-270,330-360$帧的折线结果较符合真实情况,但是在$0-30,90-150,270-300$帧间,橙色折线不能很好的刻画相机的移动路线}
        \label{fig:naive_motion}
        \end{center}
    \end{figure}
  
   该方法较上一小节的方法而言,能够比较好地解决场景切换的问题,取景框不再是永远固定在某个特定的位置,而是可以随着相机的移动,以相对平滑的方式进行切换。但是,问题依然存在。以固定间隔取关键帧的方式依然不太合理。该算法存在着这样的假设:处于两个关键帧中间的帧在原始相机路径中以比较相似的方式进行变换。关键帧间隔取得太短,会使得视频的抖动依然难以去除;而关键帧间隔太长,很可能会出现相机移动的路径与假设的路径大相径庭。譬如后一个关键帧对应的相机位置在前一个关键帧的相机位置的正右方,且相机朝向一致角度一致,在该算法中,假设了在这两个关键帧之间,相机一直是向正右方移动的。但是,存在着相机的真实路径是先向正左方移动,再折回向正右方移动的情况。在这样的情况下,该算法规划的相机路径不能很好地刻画真实的相机路径,从而导致出现一些缺省画面的情况。
   
\subsection{基于动态规划的关键帧选取}
    为了解决上一小节中提到的问题,我最终采用了动态规划的方法来选取关键帧。要采用动态规划的方法解决该相机路径优化的问题,需要先设计对每一种关键点选取方案的代价函数。 \\
    
    我设定动态规划的优化目标为最小化每一帧的相机真实位置与取景框的“偏离程度”。对于每一帧的“偏离程度”,我选择用水平方向位移$\Delta x$,垂直方向位移$\Delta y$,以及旋转的角度$\Delta\alpha$来刻画,具体“偏离程度”定义为:
    $$Diff(t)=\Delta x^2(t)+\Delta y^2(t)+w_\alpha\Delta\alpha^2(t)\eqno{(5)}$$
    $$\Delta x(t)=(T_t)_x-(F_tF_{t-1}\cdots F_{p})_x$$
    $$\Delta y(t)=(T_t)_y-(F_tF_{t-1}\cdots F_{p})_y$$
    $$\Delta x(t)=(T_t)_\alpha-(F_tF_{t-1}\cdots F_{p})_\alpha$$
    其中,$(A)_x,(A)_y,(A)_\alpha$分别表示取变换矩阵$A$的$x$分量,$y$分量以及$\alpha$角度分量,$p$为出现在第$t$帧之前的最后一个关键帧。考虑到角度$\Delta\alpha$的变化范围在$[-\pi,\pi]$之间,小的角度变化也可能造成较大的差异,所以角度分量在$Diff(t)$中应该乘以一个大于1的权值$(w_\alpha>1)$,我将$w_\alpha$设为了5(具体的权值没有仔细进行调整)。我将变换矩阵都简化成了只包含旋转和位移变换的形式,则所有变换矩阵满足$$A = \begin{pmatrix}cos(\alpha) & -sin(\alpha) &0\\sin(\alpha)&cos(\alpha)&0\\dx&dy&1\end{pmatrix}$$
    即从变换矩阵可以方便的求出$x$分量,$y$分量以及$\alpha$角度分量。\\
    
    设计好每一帧的“偏离程度”计算函数后,整个关键帧选取方案的代价$Cost(1..n)$应为中间每一帧“偏离程度”的总和。但是,直接让所有帧的“偏离程度”总和作为评判整个选取方案的好坏的唯一指标且不加任何限制条件的话,结果必然是每一帧都被选择作为关键帧,这样一来完全没有达到视频防抖的效果。\\
    
    要使得整个输出视频较为平滑,则希望关键点的选择尽量越少越好,相邻关键点之间的间隔越大越好。换句话说,有的时候,我们可以牺牲一点“偏离程度”,来换取更平滑的视觉效果。最终,我将方案的代价$Cost(1..n)$设计为:
    $$Cost(1..n)=\sum\limits_{t=1}^n\frac{Diff(t)}{(q-p+1)^2}\eqno{(6)}$$
    其中,$p$为在$t$之前的最后一个关键帧,$q$为在$t$之后的第一个关键帧。以此来达到使得相邻的关键点之间的间隔尽量大的目的。\\
    
    另外,为了使得两个关键帧之间的距离不要太近,还需要对关键帧的距离设置一个下限$lb$,选取关键帧时要求每两个关键帧的距离必须大于$lb$。\\

    至此,对于确定的关键帧选取方案,已经能够计算出整个方案的代价。对于所有的关键帧选取方案,只需要在其中选出代价最小的方案作为最终方案即可。但倘若暴力的枚举每一个关键帧的位置,枚举每一个关键帧选取方案的话,时间上的开销是难以承受的。所以,这里即需要采用动态规划的思想,将原问题分解为较小规模的子问题进行求解。\\
    
    设计动态规划函数$DP(i)$表示取第$i$帧作为关键帧的前提下,$Cost(1..t)$的大小。容易从等式$(6)$推广获得: $$Cost(i..j)=\sum\limits_{t=i}^j\frac{Diff(t)}{(q-p+1)^2}\eqno{(7)}$$
    所以,要求$DP(i)$的值,需要通过枚举确定上一个关键帧。动态规划转移方程可表示为:
    $$DP(i)=\min\limits_{1\leq j\leq i-lb}\left\{ DP(j)+Cost(i..j)\right\}\eqno{(8)}$$
    通过该状态转移方程,则可以在$O(n^2T(Cost(i..j))$的时间内获得最佳的关键帧选取方案。在确定$i,j$为关键帧后,$Cost(i..j)$需要对$i,j$之间的每一帧$t$计算$Diff(t)$,计算单帧的$Diff(t)$可在常数时间内完成。所以整个动态规划算法的时间复杂度为$O(n^3)$,还算是一个比较高效的算法。\\
    
\section{缺省画面处理算法}
    在完成对相机路径进行规划的工作以后,我发现为了使得视频的抖动减少,优化后的大部分画面与取景框不完全相同,导致了取景框内出现了缺省的像素点。在未经处理时,这些像素点灰度值都为0。在观影时,虽然画面内容的抖动变小了,但是画面的边框一直在抖动,对观影效果有一定程度的影响。所以,要使得视频防抖的输出视频效果较好,还需要对画面中的缺省值进行处理。我尝试了两种比较简单的方法。
\subsection{缩小取景框的方法}
    我尝试的第一个方法,也是非常简单的一个方法是缩小取景框的大小,如图\ref{fig:resize}。在规划的相机路径与真实相机路径高度相似的情况下,图像边缘缺省像素并不多,缩小取景框的处理方法是简单有效的。\\
    \begin{figure}[!hbt]
        \begin{center}
        \includegraphics[width=\columnwidth]{resize.png}
        \caption{缩小取景框的大小以减少画面边框的抖动。左图是经过路径规划后的输出,有“边缘黑框”情况,右图是经过缩小取景框操作后的输出}
        \label{fig:resize}
        \end{center}
    \end{figure}

    但是,这个方法也存在着两大问题。第一,当路径规划后的取景框与原始相机位置相差较大的时候,缺省像素点较多,这时候即使取景框有一定程度的缩小,但依然会存在缺省像素点的问题,所以这个问题并没有从根源得到解决。第二,被视频中裁剪掉的部分有可能有一些关键的信息,被裁剪后也会影响观影效果,如图\ref{fig:resize_fail}。
    \begin{figure}[!hbt]
        \begin{center}
        \includegraphics[width=\columnwidth]{resize_fail.png}
        \caption{左图为原始图像,右图为防抖后缩小取景框的效果。右图中小孩的头在画面中被裁剪掉了,这是一个图像中较为关键的部分,对这部分的裁剪使得整个画面的观影效果受到了影响。}
        \label{fig:resize_fail}
        \end{center}
    \end{figure}

\subsection{采用拼接相邻若干帧补全缺省部分的方法}
    裁剪掉画面边缘的做法不够理想,受到求全景图的方法的启发,我尝试用当前帧的前后$m$帧图像拼接成全景图后对当前帧缺省部分进行填充。
    这样的方法在缺省内容是视频的背景或固定不动的物体时,效果较好。图\ref{fig:padding_good}是一个平均图像补全缺省部分比较好的结果,在这一帧当中,一方面,需要补全的缺省部分比较少;另一方面,缺省部分基本上为背景的草丛、草地,并没有太大地受到移动物体(如图像中的人)的影响,所以使用相邻若干帧的平均图像可以比较好地还原背景信息。\\
    \begin{figure}[!hbt]
        \begin{center}
        \includegraphics[width=\columnwidth]{padding_good.png}
        \caption{拼接相邻若干帧补全缺省信息的方法有效的例子。左图为经过防抖优化后的原始成功图像,右图为防抖后补全缺省部分的效果}
        \label{fig:padding_good}
        \end{center}
    \end{figure}
    
    但是当缺省内容包含移动的物体时,结果就不是那么令人满意。由于在每一帧与当前帧进行对齐的时候,大部分有效的特征点都是背景点,所以对移动物体并不能很好地对齐,这使得拼接的结果往往无法很好地还原移动物体的位置和形状。图\ref{fig:padding_fail}就是一个不理想的结果。可以看到,图像下方的缺省部分草坪,基本上得到了很好的还原。图像左边的缺省信息补全的结果则略显凌乱。其实仔细观察可以发现,对于不移动的物体(凳子)来说,基本上也得到了很好的还原。但是补全的图像中,左方出现了一个人,这个人在该帧的原始图像中是没有出现的,而且这个人在相邻的若干帧图像中,是处于移动状态的。所以在拼接的时候,出现了一个没有完全很好的拼接的人。一方面,这通过前后若干帧图像还原了一部分当前帧的缺省信息;但是另一方面,这样的缺省信息的补全却是不完美的,这样别扭不自然的拼接在观影中会造成一定程度的干扰。这样的缺陷是由全景图拼接本身的局限性所带来的。\\
    \begin{figure}[!hbt]
        \begin{center}
        \includegraphics[width=\columnwidth]{padding_fail.png}
        \caption{拼接相邻若干帧补全缺省信息的方法失效的例子。左图为经过防抖优化后的原始图像,右图为防抖后补全缺省部分的效果}
        \label{fig:padding_fail}
        \end{center}
    \end{figure}
    
    我认为,想要解决这个补全缺省信息的不完美的问题,可以原来使用整幅图像进行拼接的方法不一样的是,先将相邻帧的原始图像分割成若干小块,然后对每一小块与当前帧做一个匹配和对齐,从而补全缺省信息。因为这样的方法在匹配时更注重每一小块的局部位置而非整个图像的全局位置,所以对于移动的物体来说,这样的方法更侧重于对每一小块都拼接出更好的结果。但是这样的做法也有其弊端,一方面是对每一小块进行对齐和匹配必然大大增加了处理时的时间开销,另一方面块大小的选择也是一个难题,每一块太大会使得计算结果对小的移动物体不友好,每一块太小会使得相似的块太多,难以获得较准确的对齐。上述的方法在操作上较为简单、方便。如果要追求更好的图片接合效果,可以采用Matsushita Y在Full-frame Video Stabilization\cite{full_frame}中提到的补全视频帧的方法对缺省信息进行补全。
    
    
\section{实验结果}
    我主要使用了老师提供的Dataset中的两个视频进行实验:(1)多人在花园中嬉戏的视频和(2)雪地中不知名动物在奔跑的视频。并设参数为$w_\alpha =10,lb=10$。 \\
    
    对于第(1)个视频,视频内容主要分为了两个场景。在第一部分,画面基本相似,帧与帧之间有抖动,在这部分基本上可以认为理想画面位置是基本恒定不变的。而在第二部分,画面随着人在地上滚的路径移动,在这一部分,假定理想的相机状态是以恒定的速度向画面的右下方移动的。动态规划计算得到的关键帧为第$1$帧,第$285$帧,第$376$帧,第$389$帧,其中第$1$帧和第$389$帧为起始帧和末尾帧。视频在前285帧相机基本都在同一个场景中,在随后的帧中,蓝黑色衣服的人在地上滚动,相机追随着他的移动。所以该结果与期望基本一致。第$376$帧会被选为关键帧是因为在视频的末尾,相机有停止运动的动作。\\
    
    也可以从$x,y,\alpha$三个分量上的变换来分析最终的关键帧选取结果。从图\ref{fig:DP_motion_c}中可以看出,动态规划的结果能比较好地刻画原始视频的相机路线。由于变换中的旋转分量$\alpha$数值较小,所以在该视频上关键点的选取方案对旋转角度$\alpha$不敏感。但是,使用上述路径规划方法,会使得每两个相邻帧之间都会有微小角度的旋转,在最终结果中,这样的小旋转对观影也是一种干扰。要减轻这样的干扰,我认为有两种可能可行的方法。第一种方法是先对视频进行预处理,先将类似噪声的相机旋转抖动在预处理阶段做一定的抑制。第二种方法考虑到在一般的视频中,使用者只有在进行场景切换路径变换的时候才会希望相机旋转(在此暂时不考虑为了艺术效果的旋转拍摄)。所以可以在规划的线性路径中,取一个平均的角度,设定相机在线性路径上移动时,角度不发生变化。在需要进行线性路径切换的时候,设定一个缓冲区,在缓冲区内相机进行均匀的旋转操作,实现从上一个线性路径的角度到下一个线性路径的角度的变换。但是由于时间的关系,我没有对这样的优化进行实现和实验。\\
    \begin{figure}[!hbt]
      \begin{center}
        \subfigure[$x$分量]{
          \begin{minipage}[b]{0.45\textwidth}
          \includegraphics[width=\textwidth]{DP_motionx.png}
          \end{minipage}}
        \subfigure[$y$分量]{
          \begin{minipage}[b]{0.45\textwidth}
          \includegraphics[width=\textwidth]{DP_motiony.jpg}
          \end{minipage}}
        \subfigure[$\alpha$分量]{
          \begin{minipage}[b]{0.45\textwidth}
          \includegraphics[width=\textwidth]{DP_motionA.png}
          \end{minipage}}
          \caption{视频(1)相机移动路线与动态规划移动结果(蓝色折线是原始的相机路径,橙色折线是经动态规划优化后的相机路径)}
          \label{fig:DP_motion_c}
      \end{center}
    \end{figure}
    
    对于第(2)个视频,主要分为三个场景。第一个场景中,画面基本保持稳定不变。第二个场景中,抖动幅度较大。第三个场景中,画面追踪“不知名的动物”整个相机向镜头的右下方移动。
    
\section{结论}


\begin{thebibliography}{5}

    \bibitem{GYR1}
Karpenko A, Jacobs D, Baek J, et al. Digital video stabilization and rolling shutter correction using gyroscopes[J]. CSTR, 2011, 1: 2.

    \bibitem{GYR2}
    Hanning G, Forslöw N, Forssén P E, et al. Stabilizing cell phone video using inertial measurement sensors[C]//Computer Vision Workshops (ICCV Workshops), 2011 IEEE International Conference on. IEEE, 2011: 1-8.
    
    \bibitem{L1Opt}
    Grundmann M, Kwatra V, Essa I. Auto-directed video stabilization with robust l1 optimal camera paths[C]//Computer Vision and Pattern Recognition (CVPR), 2011 IEEE Conference on. IEEE, 2011: 225-232.
    
    \bibitem{FLiu1}
    Liu F, Gleicher M, Jin H, et al. Content-preserving warps for 3D video stabilization[C]//ACM Transactions on Graphics (TOG). ACM, 2009, 28(3): 44.
    \bibitem{FLiu2}
    Liu F, Gleicher M, Wang J, et al. Subspace video stabilization[J]. ACM Transactions on Graphics (TOG), 2011, 30(1): 4.
    
    \bibitem{Matlab}
    Video Stabilization Using Point Feature Matching By Matlab.\url{http://cn.mathworks.com/help/vision/examples/video-stabilization-using-point-feature-matching.html}
    
    \bibitem{full_frame}
    Matsushita Y, Ofek E, Tang X, et al. Full-frame video stabilization[C]//Computer Vision and Pattern Recognition, 2005. CVPR 2005. IEEE Computer Society Conference on. IEEE, 2005, 1: 50-57.
    

\end{thebibliography}

% Your document ends here!
\end{document}