\documentclass[journal, a4paper]{IEEEtran}

% some very useful LaTeX packages include:

%\usepackage{cite}      % Written by Donald Arseneau
                        % V1.6 and later of IEEEtran pre-defines the format
                        % of the cite.sty package \cite{} output to follow
                        % that of IEEE. Loading the cite package will
                        % result in citation numbers being automatically
                        % sorted and properly "ranged". i.e.,
                        % [1], [9], [2], [7], [5], [6]
                        % (without using cite.sty)
                        % will become:
                        % [1], [2], [5]--[7], [9] (using cite.sty)
                        % cite.sty's \cite will automatically add leading
                        % space, if needed. Use cite.sty's noadjust option
                        % (cite.sty V3.8 and later) if you want to turn this
                        % off. cite.sty is already installed on most LaTeX
                        % systems. The latest version can be obtained at:
                        % http://www.ctan.org/tex-archive/macros/latex/contrib/supported/cite/

\usepackage{graphicx}   % Written by David Carlisle and Sebastian Rahtz
                        % Required if you want graphics, photos, etc.
                        % graphicx.sty is already installed on most LaTeX
                        % systems. The latest version and documentation can
                        % be obtained at:
                        % http://www.ctan.org/tex-archive/macros/latex/required/graphics/
                        % Another good source of documentation is "Using
                        % Imported Graphics in LaTeX2e" by Keith Reckdahl
                        % which can be found as esplatex.ps and epslatex.pdf
                        % at: http://www.ctan.org/tex-archive/info/

%\usepackage{psfrag}    % Written by Craig Barratt, Michael C. Grant,
                        % and David Carlisle
                        % This package allows you to substitute LaTeX
                        % commands for text in imported EPS graphic files.
                        % In this way, LaTeX symbols can be placed into
                        % graphics that have been generated by other
                        % applications. You must use latex->dvips->ps2pdf
                        % workflow (not direct pdf output from pdflatex) if
                        % you wish to use this capability because it works
                        % via some PostScript tricks. Alternatively, the
                        % graphics could be processed as separate files via
                        % psfrag and dvips, then converted to PDF for
                        % inclusion in the main file which uses pdflatex.
                        % Docs are in "The PSfrag System" by Michael C. Grant
                        % and David Carlisle. There is also some information
                        % about using psfrag in "Using Imported Graphics in
                        % LaTeX2e" by Keith Reckdahl which documents the
                        % graphicx package (see above). The psfrag package
                        % and documentation can be obtained at:
                        % http://www.ctan.org/tex-archive/macros/latex/contrib/supported/psfrag/

%\usepackage{subfigure} % Written by Steven Douglas Cochran
                        % This package makes it easy to put subfigures
                        % in your figures. i.e., "figure 1a and 1b"
                        % Docs are in "Using Imported Graphics in LaTeX2e"
                        % by Keith Reckdahl which also documents the graphicx
                        % package (see above). subfigure.sty is already
                        % installed on most LaTeX systems. The latest version
                        % and documentation can be obtained at:
                        % http://www.ctan.org/tex-archive/macros/latex/contrib/supported/subfigure/

\usepackage{url}        % Written by Donald Arseneau
                        % Provides better support for handling and breaking
                        % URLs. url.sty is already installed on most LaTeX
                        % systems. The latest version can be obtained at:
                        % http://www.ctan.org/tex-archive/macros/latex/contrib/other/misc/
                        % Read the url.sty source comments for usage information.

%\usepackage{stfloats}  % Written by Sigitas Tolusis
                        % Gives LaTeX2e the ability to do double column
                        % floats at the bottom of the page as well as the top.
                        % (e.g., "\begin{figure*}[!b]" is not normally
                        % possible in LaTeX2e). This is an invasive package
                        % which rewrites many portions of the LaTeX2e output
                        % routines. It may not work with other packages that
                        % modify the LaTeX2e output routine and/or with other
                        % versions of LaTeX. The latest version and
                        % documentation can be obtained at:
                        % http://www.ctan.org/tex-archive/macros/latex/contrib/supported/sttools/
                        % Documentation is contained in the stfloats.sty
                        % comments as well as in the presfull.pdf file.
                        % Do not use the stfloats baselinefloat ability as
                        % IEEE does not allow \baselineskip to stretch.
                        % Authors submitting work to the IEEE should note
                        % that IEEE rarely uses double column equations and
                        % that authors should try to avoid such use.
                        % Do not be tempted to use the cuted.sty or
                        % midfloat.sty package (by the same author) as IEEE
                        % does not format its papers in such ways.

\usepackage{amsmath}    % From the American Mathematical Society
                        % A popular package that provides many helpful commands
                        % for dealing with mathematics. Note that the AMSmath
                        % package sets \interdisplaylinepenalty to 10000 thus
                        % preventing page breaks from occurring within multiline
                        % equations. Use:
%\interdisplaylinepenalty=2500
                        % after loading amsmath to restore such page breaks
                        % as IEEEtran.cls normally does. amsmath.sty is already
                        % installed on most LaTeX systems. The latest version
                        % and documentation can be obtained at:
                        % http://www.ctan.org/tex-archive/macros/latex/required/amslatex/math/



% Other popular packages for formatting tables and equations include:

%\usepackage{array}
% Frank Mittelbach's and David Carlisle's array.sty which improves the
% LaTeX2e array and tabular environments to provide better appearances and
% additional user controls. array.sty is already installed on most systems.
% The latest version and documentation can be obtained at:
% http://www.ctan.org/tex-archive/macros/latex/required/tools/

% V1.6 of IEEEtran contains the IEEEeqnarray family of commands that can
% be used to generate multiline equations as well as matrices, tables, etc.

\usepackage{graphicx}
\usepackage{subfigure}
\usepackage{wrapfig}
\usepackage{xeCJK}
\setCJKmainfont[
  BoldFont=WenQuanYi Zen Hei,
  ItalicFont=AR PL KaitiM GB]
  {AR PL SungtiL GB}
\setCJKsansfont{WenQuanYi Zen Hei}
\setCJKmonofont{cwTeXFangSong}

% Also of notable interest:
% Scott Pakin's eqparbox package for creating (automatically sized) equal
% width boxes. Available:
% http://www.ctan.org/tex-archive/macros/latex/contrib/supported/eqparbox/

% *** Do not adjust lengths that control margins, column widths, etc. ***
% *** Do not use packages that alter fonts (such as pslatex).         ***
% There should be no need to do such things with IEEEtran.cls V1.6 and later.


% Your document starts here!
\begin{document}

% Define document title and author
    \title{基于特征匹配与动态规划的视频防抖算法}
    \author{刘璐 13307130520}
    \maketitle

% Each section begins with a \section{title} command
\section{引言}
    随着摄影录像技术的发展,越来越多的人选择使用小型手持设备进行视频录制,记录下一些难忘的时刻。大部分使用小型手持设备进行视频录制的使用者在录制过程中一方面缺乏固定镜头的设备,另一方面没有专业的录像技术,以致于使用小型手持设备录制的视频都会出现一定的画面抖动,在一定程度上影响了观影效果。视频防抖技术的出现即是为了解决这一难题。 \\
    
    要减少视频中影响视频质量的抖动现象,一种方法是使用手持设备的陀螺仪、加速度仪,记录视频拍摄时设备在三维空间三个角度的旋转以及加速度,从设备的移动路径反向恢复视频的质量。这样的方法能获取录制设备的移动轨迹,从而提高视频处理的效果\cite{GYR1}\cite{GYR2}。但是,一方面,对于网络上大部分已存在的视频来说,录制时的设备移动记录已经无法获取;另一方面,不同的设备内置的陀螺仪、加速度仪会有不同的误差,要设计对于所有设备都行之有效的算法有一定的难度。\\
    
    另一种方法是,直接对视频进行处理。传统的视频防抖方法都是基于2D图像进行处理的。一般的处理方法为对相邻的帧进行特征点匹配,从帧与帧之间的对齐关系获得录像设备视角的移动,再对视角的移动路线进行平滑处理,使得输出的视频移动得相对稳定\cite{L1Opt}。也有新兴的技术通过视频重构三维模型,来实现视频防抖。该技术重构像素点的3D点云,以及录影设备在三维空间上的移动路线,对设备移动路线进行平滑处理后,利用3D点云恢复视频图像\cite{FLiu1}\cite{FLiu2}。 \\
    
    我的方法主要是基于2D图像进行处理的。在对视角移动线路进行平滑处理时,只考虑线性的移动,利用若干条线段拟合相机的移动路线。在拟合过程中,选定若干关键帧为不动帧,从一个关键帧到另一个关键帧的过程中,相机路径始终采用相同的移动。在关键帧选取的关键环节,采用动态规划算法,最优化关键帧的选取,从而获得尽量与原始相机移动路径相似的线性移动路径,实现视频的防抖。

\section{视频防抖算法}
\subsection{固定视角的视频防抖}
    刚开始的时候,我采用了Matlab手册中基于特征点匹配的视频防抖方法\cite{Matlab}。首先,对于视频中的每一帧图像$I_{1}, I_{2}, \cdots, I_{n}$,求SURF特征点。然后,对视频中相邻两帧的图像$(I_{t}, I_{t+1})$进行特征点匹配,获得相邻两帧的变换矩阵$F_{t}$。相机路径$C_{t}$可以与前一帧的相机路径$C_{t-1}$以及变换矩阵$F_{t}$相关:$C_{t}=C_{t-1}*F_{t}$,且对$C_{t}$进行迭代计算可得到$C_{t}=F_{t}F_{t-1}\cdots F_{1}$。 \\
    
    进行视频防抖最直接的想法就是:对每一帧原始图像,使用$C_{t}$矩阵进行变换,这样一来画面中的大部分特征点都会基本保持静止,从而实现了减少视频抖动的目的。这样的方法在固定视角、单一场景的视频中的确能取得较好的效果。但是,在一个完整视频中,往往会有相机的移动、场景的切换。因为该方法的所有帧都是向第一帧对齐的,后面的帧中特征点基本与第一帧中特征点的位置吻合,所以在遇到相机移动和场景切换的时候,相机的拍摄内容已经移出或部分移出第一帧的取景框内,这个时候在第一帧的取景框内的内容即为缺省值。因此,该方法对有场景切换的视频是基本无效的, 如图。
    
\subsection{选取固定关键帧的视频防抖}
    考虑到上述方法失效的原因主要是原始相机路径的移动与输出视频的取景框固定不动的矛盾。所以,我开始寻求让取景框动起来的视频防抖方法。 \\
    
    在这个过程中,我读到了M. Grundmann提出的利用L1范数线性优化相机路径的方法\cite{L1Opt}。在该论文中提到,在专业的视频拍摄中,相机的移动路线$P(t)$主要分为以下三种:
    \begin{itemize}
    \item 固定不动的相机路线,$DP(t)=0$
    \item 以恒定速度移动的相机路线,$DP^2(t)=0$
    \item 以恒定加速度移动的相机路线,$DP^3(t)=0$
    \end{itemize}
    在该论文中,M. Grundmann旨在用上述三种相机移动路径来近似原始的相机路径,使得输出的视频结果即保持了原来的视频内容,又实现了防抖的效果。在该论文中,M. Grundmann最终采用了L1范数线性规划的方法进行路径优化。\\
    
    固定不动的相机路线的视频防抖问题在上一小节中已经完美解决。我认为,长时间以恒定加速度移动的相机路线只会在有专业的控制相机移动路线的设备上才有可能获得。一般在使用手持设备进行拍摄的条件下,使用者在移动相机时的目的在于获得恒定速度移动的相机路线,在这种情况下,恒定加速度移动的相机路线只会在使用者希望相机改变移动速度的短暂时刻出现。所以,我决定只采用恒定速度移动的相机路线去近似原始输入视频的相机路线。\\
    
   一开始,我采用了固定间隔的关键帧选取,如图\ref{fig:naive_motion}。即每$k(k=30)$帧选一帧作为关键帧,关键帧的相机路径与路径规划后的相机路径$P_t$吻合,$P_t=C_t$,并假设取景框在关键帧$C_{ki}$与$C_{ki+k}$之间以恒定的速度移动,即变换矩阵相同,
   $$H_{ki+1}=H_{ki+2}=\cdots=H_{ki+k}\eqno{(1)}$$同时由于关键帧的性质,应有
   $$H_{ki+1}*H_{ki+2}*\cdots*H_{ki+k}=F_{ki+1}*F_{ki+2}*\cdots*F_{ki+k}\eqno{(2)}$$
   由等式($1$)($2$)得:$$H_{ki+j}=(F_{ki+1}*F_{ki+2}*\cdots*F_{ki+k})^{\frac{1}{k}}\eqno{(3)}$$从而可以求得每一帧图像的变换矩阵
   $$T_{ki+j}=H_{ki+j}^{-j}*F_{ki+1}*F_{ki+2}*\cdots*F_{ki+j}\eqno{(4)}$$
   这样一来,以固定间隔取关键帧,关键帧间的相邻帧的取景窗以相同的变换矩阵进行变换,就实现了在两个关键帧之间的画面以比较平滑的方式进行移动、变换。\\
   \begin{figure}[!hbt]
        \begin{center}
        \includegraphics[width=\columnwidth]{naive_motion.png}
        \caption{蓝色的折线是视频原始的相机在$x$方向的移动,橙色的折线是每$k(k=30)$帧取一帧作为关键帧,用关键帧之间的折线去近似相机移动的结果。可以看到,在第$30-60,150-270,330-360$帧的折线结果较符合真实情况,但是在$0-30,90-150,270-300$帧间,橙色折线不能很好的刻画相机的移动路线}
        \label{fig:naive_motion}
        \end{center}
    \end{figure}
  
   该方法较上一小节的方法而言,能够比较好地解决场景切换的问题,取景框不再是永远固定在某个特定的位置,而是可以随着相机的移动,以相对平滑的方式进行切换。但是,问题依然存在。以固定间隔取关键帧的方式依然不太合理。该算法存在着这样的假设:处于两个关键帧中间的帧在原始相机路径中以比较相似的方式进行变换。关键帧间隔取得太短,会使得视频的抖动依然难以去除;而关键帧间隔太长,很可能会出现相机移动的路径与假设的路径大相径庭。譬如后一个关键帧对应的相机位置在前一个关键帧的相机位置的正右方,且相机朝向一致角度一致,在该算法中,假设了在这两个关键帧之间,相机一直是向正右方移动的。但是,存在着相机的真实路径是先向正左方移动,再折回向正右方移动的情况。在这样的情况下,该算法规划的相机路径不能很好地刻画真实的相机路径,从而导致出现一些缺省画面的情况。
   
\subsection{基于动态规划的关键帧选取}
    为了解决上一小节中提到的问题,我最终采用了动态规划的方法来选取关键帧。要采用动态规划的方法解决该相机路径优化的问题,需要先设计对每一种关键点选取方案的代价函数。 \\
    
    我设定动态规划的优化目标为最小化每一帧的相机真实位置与取景框的“偏离程度”。对于每一帧的“偏离程度”,我选择用水平方向位移$\Delta x$,垂直方向位移$\Delta y$,以及旋转的角度$\Delta\alpha$来刻画,具体“偏离程度”定义为:
    $$Diff(t)=\Delta x^2(t)+\Delta y^2(t)+w_\alpha\Delta\alpha^2(t)\eqno{(5)}$$
    $$\Delta x(t)=(T_t)_x-(F_tF_{t-1}\cdots F_{p})_x$$
    $$\Delta y(t)=(T_t)_y-(F_tF_{t-1}\cdots F_{p})_y$$
    $$\Delta x(t)=(T_t)_\alpha-(F_tF_{t-1}\cdots F_{p})_\alpha$$
    其中,$(A)_x,(A)_y,(A)_\alpha$分别表示取变换矩阵$A$的$x$分量,$y$分量以及$\alpha$角度分量,$p$为出现在第$t$帧之前的最后一个关键帧。考虑到角度$\Delta\alpha$的变化范围在$[-\pi,\pi]$之间,小的角度变化也可能造成较大的差异,所以角度分量在$Diff(t)$中应该乘以一个大于1的权值$(w_\alpha>1)$,我将$w_\alpha$设为了5(具体的权值没有仔细进行调整)。我将变换矩阵都简化成了只包含旋转和位移变换的形式,则所有变换矩阵满足$$A = \begin{pmatrix}cos(\alpha) & -sin(\alpha) &0\\sin(\alpha)&cos(\alpha)&0\\dx&dy&1\end{pmatrix}$$
    即从变换矩阵可以方便的求出$x$分量,$y$分量以及$\alpha$角度分量。\\
    
    设计好每一帧的“偏离程度”计算函数后,整个关键帧选取方案的代价$Cost(1..n)$应为中间每一帧“偏离程度”的总和。但是,直接让所有帧的“偏离程度”总和作为评判整个选取方案的好坏的唯一指标且不加任何限制条件的话,结果必然是每一帧都被选择作为关键帧,这样一来完全没有达到视频防抖的效果。\\
    
    要使得整个输出视频较为平滑,则希望关键点的选择尽量越少越好,相邻关键点之间的间隔越大越好。换句话说,有的时候,我们可以牺牲一点“偏离程度”,来换取更平滑的视觉效果。最终,我将方案的代价$Cost(1..n)$设计为:
    $$Cost(1..n)=\sum\limits_{t=1}^n\frac{Diff(t)}{(q-p+1)^2}\eqno{(6)}$$
    其中,$p$为在$t$之前的最后一个关键帧,$q$为在$t$之后的第一个关键帧。以此来达到使得相邻的关键点之间的间隔尽量大的目的。\\
    
    另外,为了使得两个关键帧之间的距离不要太近,还需要对关键帧的距离设置一个下限$lb$,选取关键帧时要求每两个关键帧的距离必须大于$lb$。\\

    至此,对于确定的关键帧选取方案,已经能够计算出整个方案的代价。对于所有的关键帧选取方案,只需要在其中选出代价最小的方案作为最终方案即可。但倘若暴力的枚举每一个关键帧的位置,枚举每一个关键帧选取方案的话,时间上的开销是难以承受的。所以,这里即需要采用动态规划的思想,将原问题分解为较小规模的子问题进行求解。\\
    
    设计动态规划函数$DP(i)$表示取第$i$帧作为关键帧的前提下,$Cost(1..t)$的大小。容易从等式$(6)$推广获得: $$Cost(i..j)=\sum\limits_{t=i}^j\frac{Diff(t)}{(q-p+1)^2}\eqno{(7)}$$
    所以,要求$DP(i)$的值,需要通过枚举确定上一个关键帧。动态规划转移方程可表示为:
    $$DP(i)=\min\limits_{1\leq j\leq i-lb}\left\{ DP(j)+Cost(i..j)\right\}\eqno{(8)}$$
    通过该状态转移方程,则可以在$O(n^2T(Cost(i..j))$的时间内获得最佳的关键帧选取方案。在确定$i,j$为关键帧后,$Cost(i..j)$需要对$i,j$之间的每一帧$t$计算$Diff(t)$,计算单帧的$Diff(t)$可在常数时间内完成。所以整个动态规划算法的时间复杂度为$O(n^3)$,还算是一个比较高效的算法。\\
    
\section{缺省画面处理算法}
    在完成对相机路径进行规划的工作以后,我发现为了使得视频的抖动减少,优化后的大部分画面与取景框不完全相同,导致了取景框内出现了缺省的像素点。在未经处理时,这些像素点灰度值都为0。在观影时,虽然画面内容的抖动变小了,但是画面的边框一直在抖动,对观影效果有一定程度的影响。所以,要使得视频防抖的输出视频效果较好,还需要对画面中的缺省值进行处理。我尝试了两种比较简单的方法。
\subsection{缩小取景框的方法}
    我尝试的第一个方法,也是非常简单的一个方法是缩小取景框的大小,如图\ref{fig:resize}。在规划的相机路径与真实相机路径高度相似的情况下,图像边缘缺省像素并不多,缩小取景框的处理方法是简单有效的。\\
    \begin{figure}[!hbt]
        \begin{center}
        \includegraphics[width=\columnwidth]{resize.png}
        \caption{缩小取景框的大小以减少画面边框的抖动。左图是经过路径规划后的输出,有“边缘黑框”情况,右图是经过缩小取景框操作后的输出}
        \label{fig:resize}
        \end{center}
    \end{figure}

    但是,这个方法也存在着两大问题。第一,当路径规划后的取景框与原始相机位置相差较大的时候,缺省像素点较多,这时候即使取景框有一定程度的缩小,但依然会存在缺省像素点的问题,所以这个问题并没有从根源得到解决。第二,被视频中裁剪掉的部分有可能有一些关键的信息,被裁剪后也会影响观影效果,如图\ref{fig:resize_fail}。
    \begin{figure}[!hbt]
        \begin{center}
        \includegraphics[width=\columnwidth]{resize_fail.png}
        \caption{左图为原始图像,右图为防抖后缩小取景框的效果。右图中小孩的头在画面中被裁剪掉了,这是一个图像中较为关键的部分,对这部分的裁剪使得整个画面的观影效果受到了影响。}
        \label{fig:resize_fail}
        \end{center}
    \end{figure}

\subsection{采用平均图像补全缺省部分的方法}
    裁剪掉画面边缘的做法不够理想,受到求全景图的方法的启发,我尝试用当前帧的前后$m$帧图像对齐后的图像平均值对缺省部分进行填充。
    这样的方法在缺省内容是视频的背景或固定不动的物体时,效果较好。但是当缺省内容是移动的物体时,就会出现“幽灵现象”,并且难以还原原来物体的形状。
    
\section{实验结果}
    我主要使用了老师提供的Dataset中的两个视频进行实验:(1)多人在花园中嬉戏的视频和(2)雪地中不知名动物在奔跑的视频。并设参数为$w_\alpha =10,lb=10$。
    对于第(1)个视频,视频内容主要分为了两个场景。在第一部分,画面基本相似,帧与帧之间有抖动,在这部分基本上可以认为理想画面位置是基本恒定不变的。而在第二部分,画面随着人在地上滚的路径移动,在这一部分,假定理想的相机状态是以恒定的速度向画面的右下方移动的。动态规划计算得到的关键帧为第$1$帧,第$285$帧,第$376$帧,第$389$帧,其中第$1$帧和第$389$帧为起始帧和末尾帧。视频在前285帧相机基本都在同一个场景中,在随后的帧中,蓝黑色衣服的人在地上滚动,相机追随着他的移动。所以该结果与期望基本一致。第$376$帧会被选为关键帧是因为在视频的末尾,相机有停止运动的动作。\\
    
    我们也可以从$x,y,\alpha$三个分量上的变换来分析最终的关键帧选取结果。从图\ref{fig:DP_motion_c}中可以看出,动态规划的结果能比较好地刻画原始视频的相机路线。由于变换中的旋转分量$\alpha$数值较小,所以在该视频上关键点的选取方案对旋转角度$\alpha$不敏感。 \\
    \begin{figure}[!hbt]
      \begin{center}
        \subfigure[相机变换的$x$分量]{
          \begin{minipage}[b]{0.45\textwidth}
          \includegraphics[width=\textwidth]{DP_motionx.png}
          \end{minipage}}
        \subfigure[相机路径变换的$y$分量]{
          \begin{minipage}[b]{0.45\textwidth}
          \includegraphics[width=\textwidth]{DP_motiony.jpg}
          \end{minipage}}
        \subfigure[相机路径变换的$\alpha$分量]{
          \begin{minipage}[b]{0.45\textwidth}
          \includegraphics[width=\textwidth]{DP_motionA.png}
          \end{minipage}}
          \caption{相机移动路线与动态规划移动结果}
          \label{fig:DP_motion_c}
      \end{center}
    \end{figure}
    
    对于第(2)个视频
    
\section{结论}


\begin{thebibliography}{5}

    \bibitem{GYR1}
Karpenko A, Jacobs D, Baek J, et al. Digital video stabilization and rolling shutter correction using gyroscopes[J]. CSTR, 2011, 1: 2.

    \bibitem{GYR2}
    Hanning G, Forslöw N, Forssén P E, et al. Stabilizing cell phone video using inertial measurement sensors[C]//Computer Vision Workshops (ICCV Workshops), 2011 IEEE International Conference on. IEEE, 2011: 1-8.
    
    \bibitem{L1Opt}
    Grundmann M, Kwatra V, Essa I. Auto-directed video stabilization with robust l1 optimal camera paths[C]//Computer Vision and Pattern Recognition (CVPR), 2011 IEEE Conference on. IEEE, 2011: 225-232.
    
    \bibitem{FLiu1}
    Liu F, Gleicher M, Jin H, et al. Content-preserving warps for 3D video stabilization[C]//ACM Transactions on Graphics (TOG). ACM, 2009, 28(3): 44.
    \bibitem{FLiu2}
    Liu F, Gleicher M, Wang J, et al. Subspace video stabilization[J]. ACM Transactions on Graphics (TOG), 2011, 30(1): 4.
    
    \bibitem{Matlab}
    Video Stabilization Using Point Feature Matching By Matlab.\url{http://cn.mathworks.com/help/vision/examples/video-stabilization-using-point-feature-matching.html}
    

\end{thebibliography}

% Your document ends here!
\end{document}